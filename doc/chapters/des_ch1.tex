\chapter{Data Encryption Standard}

\begin{itemize}
	\item Symmetric Block Cipher.
	\item A.k.a Data Encryption Algorithm.
	\item Adopted by NIST in 1977.
	\item Advanced Encryption Standard (AES) in 2001.
\end{itemize}

\begin{table}[h!]\centering\renewcommand{\arraystretch}{1.25} % Increase row height by 1.5 times
\caption{Parameters of the Block Cipher DES}
\begin{tabular*}{\textwidth}{@{\extracolsep{\fill}}cccccc}
	\toprule[1.2pt]
	Input & Output & Master Key & Sub-key & Round Key & No. of rounds \\
	64-bit & 64-bit & 64-bit & 56-bit & 48-bit & 16 rounds \\
%	\multirow{3}{*}{Algorithms} & Block & Key & Number of & Round-Key & Number of & Total Size of\\
%	& Size & Length & Rounds &  Length & Round-Keys & Round-Keys \\
%	& ($N_b$-word) & ($N_k$-word) & ($N_r$)& (word) & ($N_r+1$)& ($N_b(N_r+1)$)\\
%	\hline\hline
%	AES-128 & 4 & 4 (4$\cdot$32-bit) & 10 & 4 & 11 & 44 (176-byte) \\
%	AES-192 & 4 & 6 (6$\cdot$32-bit) & 12 & 4 & 13 & 52 (208-byte) \\
%	AES-256 & 4 & 8 (8$\cdot$32-bit) & 14 & 4 & 15 & 60 (240-byte) \\
	\bottomrule[1.2pt]
\end{tabular*}
\end{table}

\section{Key Schedule}

\newpage
\section{The S-Boxes of DES}\begin{itemize}
	\item The Data Encryption Standard (DES) is a 64-bit block cipher with 16 rounds and a 56-bit key.
	\item There are eight S-Boxes used in DES. They are usually denoted by $S_1, \ldots, S_8$.
	\item Let $\binaryfield^n$ denote the set of sequences of bits ($\zero$'s and $\one$'s) of length $n$. We can equivalently think of $\binaryfield^n$ as the integers $\{0, 1, \ldots, 2^n - 1\}$.
	\item Each S-Box of DES is a function $S_i : \binaryfield^6 \rightarrow \binaryfield^4$.
	\item So we can think of the domain of each function $S_i$ to be the set of integers $\{0, 1, \ldots, 63\}$ and the range to be $\{0, 1, \ldots, 15\}$.
\end{itemize}

\begin{itemize}
	\item The natural way to specify any function 
	\[S : \{0, \ldots, 63\} \rightarrow \{0, \ldots, 15\}\]
	would just be as a list of 64 values, where the $i$-th value of the list for $0 \leq i \leq 63$ is the value of $S(i)$. Moreover, each value in the list would be an element of $\{0, \ldots, 15\}$.
	\item Even though the S-Boxes of DES are specified using a list of 64 values, with each value being in $\{0, \ldots, 15\}$, the list has to be interpreted differently than the natural way to determine the values of the given S-Box.
\end{itemize}

\url{https://en.wikipedia.org/wiki/DES_supplementary_material}

\begin{itemize}
	\item Let $x = (b_5b_4b_3b_2b_1b_0) \in \binaryfield^6$ be a given 6-bit input.
	\item We form a 2-bit value $r = (b_5b_0) \in \binaryfield^2$ using the most significant bit $b_5$ and the least significant bit $b_0$ of $x$.
	\item The value of $r$ is in $\{0, 1, 2, 3\}$ and it determines the row of the table.
	\item We also form a 4-bit value $c = (b_4b_3b_2b_1) \in \binaryfield^4$ using the inner four bits of $x$.
	\item The value of $c$ is in $\{0, 1, \ldots, 15\}$ and it determines the column of the table.
	\item Here we are using zero based counting. So for example $r = 0$ refers to the first row and $c = 15$ refers to the sixteenth column.
\end{itemize}

\begin{example}
If $x = (27)_{10} = (011011)_2$, the outer two bits are $r = (01)_2 = (1)_{10}$, so we look in the second row. The inner four bits are $c = (1101)_2 = (13)_{10}$ which gives the fourteenth column.

\[
\begin{aligned}
	x &= \zero\textcolor{blue}{\texttt{1101}}\one \rightarrow
	\quad \begin{cases}
		\zero\one & \rightarrow \text{second row} \\
		\textcolor{blue}{\texttt{1101}} & \rightarrow \text{fourteenth column}
	\end{cases}
\end{aligned}
\]
Therefore the value of $S_5(x)$ is $(9)_{10} = (1001)_2 \in \binaryfield^4$.
\end{example}

\newpage
\begin{itemize}
	\item 
	Let $x = (b_5b_4b_3b_2b_1b_0) \in \binaryfield^6$.
	To access the most significant bit $b_5$, we can use: 
\begin{lstlisting}[style=C]
msb = x >> 5
\end{lstlisting}
\vspace{8pt}
	To access the least significant bit $b_0$, we can use: 
\begin{lstlisting}[style=C]
lsb = x & 1
\end{lstlisting}
\vspace{8pt}
	So the two-bit value $(b_5b_0) \in \binaryfield^2$ formed by the most and least significant bits is:
\begin{lstlisting}[style=C]
row = (msb << 1) | lsb
\end{lstlisting}
\vspace{8pt}
	This gives us the row of the table we need to look at.
	\vspace{20pt}
	\item To access the inner four bits of $x$ we start by right-shifting one position to knock off the least significant bit $b_0$. What's left is $(b_5b_4b_3b_2b_1)$.
	
	To get rid of $b_5$ but keep the rest of the bits we AND this value with $(1111)_2$ which in hex is $(f)_{16}$.
	This is coded as:
\begin{lstlisting}[style=C]
col = (x >> 1) & 0xf
\end{lstlisting}
This gives us the column of the table we need to look at.
\end{itemize}

\newpage
\begin{itemize}
	\item To find the value of $S(x)$ we need to find the list index associated to $x$.
	
	Now that we know the row and column values determined by $x$ we can get the corresponding index of the table by:
	
	\[ \text{index} = \text{row} \times 16 + \text{col} \]
	
	Note that we are multiplying by 16 because each row of the table has 16 values.
	
	We can code this list index function as follows:
	
\begin{lstlisting}[style=C]
// # x is 6-bits
// # the index returned is in {0,...,63}
// def index(x):
//	msb = x >> 5 # most significant bit (the leftmost bit)
//	lsb = x & 1  # least significant bit (the rightmost bit)
//	row = (msb << 1) | lsb  # outer 2-bits of x
//	col = (x >> 1) & 0xf    # inner 4-bits of x
//	return row * 16 + col   # calculate the list index
	
fn index(x: u8) -> usize {
	// Ensure x is 6 bits
	let x = x & 0x3F; // Apply mask to limit to 6 bits if not already
	
	// Extract bits similarly to the Python version, utilizing Rust's type safety and bit operations
	let msb: u8 = x >> 5; // Most significant bit
	let lsb: u8 = x & 1;  // Least significant bit
	let row: u8 = (msb << 1) | lsb; // Combine msb and lsb to form the row
	let col: u8 = (x >> 1) & 0xF;   // Extract the 4 middle bits for the column
	
	// Calculate the final index, converting the row and col into usize for the return type
	(row as usize) * 16 + (col as usize)
}
\end{lstlisting}
\end{itemize}

\newpage
So the S-Box \(S_5\) of DES can be implemented in the following way:

\begin{lstlisting}[style=C]
// # x is 6-bits
// # The index returned is in [0,...,63]
// def index(x):
//	msb = x >> 5  # most significant bit (the leftmost bit)
//	lsb = x & 1   # least significant bit (the rightmost bit)
//	row = (msb << 1) | lsb  # outer 2-bits of x
//	col = (x >> 1) & 0xf    # inner 4-bits of x
//	return row * 16 + col   # calculate the list index

// s5 = [2, 12, 4, 1, 7, 10, 11, 6, 8, 5, 3, 15, 13, 0, 14, 9,
//	  14, 11, 2, 12, 4, 7, 13, 1, 5, 0, 15, 10, 3, 9, 8, 6,
//	  4, 2, 1, 11, 10, 13, 7, 8, 15, 9, 12, 5, 6, 3, 0, 14,
//	  11, 8, 12, 7, 1, 14, 2, 13, 6, 15, 0, 9, 10, 4, 5, 3]

// # x is 6-bits,
// # The return value is 4-bits
// def S5(x):
// 	return s5[index(x)]

// # Example usage
// print(S5(27))  # should get 9

const S5_TABLE: [u8; 64] = [
	2, 12, 4, 1, 7, 10, 11, 6, 8, 5, 3, 15, 13, 0, 14, 9,
	14, 11, 2, 12, 4, 7, 13, 1, 5, 0, 15, 10, 3, 9, 8, 6,
	4, 2, 1, 11, 10, 13, 7, 8, 15, 9, 12, 5, 6, 3, 0, 14,
	11, 8, 12, 7, 1, 14, 2, 13, 6, 15, 0, 9, 10, 4, 5, 3,
];

fn index(x: u8) -> usize {
	let x = x & 0x3F; // Ensure x is 6 bits
	let msb = x >> 5; // Most significant bit
	let lsb = x & 1;  // Least significant bit
	let row = (msb << 1) | lsb; // Outer 2-bits of x
	let col = (x >> 1) & 0xF;   // Inner 4-bits of x
	(row as usize) * 16 + (col as usize) // Calculate the list index
}

fn s5(x: u8) -> u8 {
	// Call index(x) to get the index, then use it to retrieve the value from S5_TABLE
	S5_TABLE[index(x)]
}

fn main() {
	println!("{}", s5(27)); // Should output 9
}
\end{lstlisting}





